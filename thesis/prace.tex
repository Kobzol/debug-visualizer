% Nejprve uvedeme tridu dokumentu s volbami
%\documentclass[bc,male,java,dept460]{diploma}						% jednostranny dokument
\documentclass[bc,male,java,dept460,twoside]{diploma}		% oboustranny dokument
\usepackage[czech]{babel}


% Zadame pozadovane vstupy pro generovani titulnich stran.
\ThesisAuthor{Jiří Dvorský}

% U bakalarske praxe neni nutne nazev zadavat
\ThesisTitle{Ukázka sazby diplomové nebo bakalářské práce}

% U bakalarske praxe neni nutne anglicky nazev zadavat
\EnglishThesisTitle{Diploma Thesis Typesetting Demo}

\SubmissionDate{16. dubna 2009}

\PrintPublicationAgreement{true}

\AccessRestriction{Zde vložte text dohodnutého omezení přístupu k Vaší práci, chránící například firemní know-how.
Zde vložte text dohodnutého omezení přístupu k Vaší práce, chránící například firemní know-how.
A zavazujete se, že\par
\begin{enumerate}
\item o práci nikomu neřeknete,
\item po obhajobě na ni zapomenete a
\item budete popírat její existenci.
\end{enumerate}
A ještě jeden důležitý odstavec. A ještě jeden důležitý odstavec.
A ještě jeden důležitý odstavec. A ještě jeden důležitý odstavec.
A ještě jeden důležitý odstavec. A ještě jeden důležitý odstavec.
Konec textu dohodnutého omezení přístupu k Vaší práci.}

\Thanks{Rád bych na tomto místě poděkovala všem, kteří mi s prací pomohli, protože bez nich by tato práce nevznikla.}

\CzechAbstract{Tohle je nějaký abstrakt. Tohle je nějaký abstrakt. Tohle je nějaký abstrakt. Tohle je nějaký abstrakt.
Tohle je nějaký abstrakt. Tohle je nějaký abstrakt. Tohle je nějaký abstrakt. Tohle je nějaký abstrakt.
Tohle je nějaký abstrakt. Tohle je nějaký abstrakt. Tohle je nějaký abstrakt. Tohle je nějaký abstrakt.}

\CzechKeywords{typografie, \LaTeX, diplomová práce}

\EnglishAbstract{This is English abstract. This is English abstract. This is English abstract. This is English abstract. This is English abstract. This is English abstract.}

\EnglishKeywords{typography, \LaTeX, master thesis}

% Pridame pouzivane zkratky (pokud nejake pouzivame).
\AddAcronym{DVD}{Digital Versatile Disc}
\AddAcronym{TNT}{Trinitrotoluen}
\AddAcronym{OASIS}{Organization For The Advancement Of Structured Information Systems}
\AddAcronym{HTML}{Hyper Text Markup Language}


% Zadame cestu a jmeno souboru ci nekolika souboru s digitalizovanou podobou zadani prace
% Pri sazbe se pak hledaji soubory Figures/Zadani1.jpg, Figures/Zadani2.jpg atd.
% Do diplomove prace se postupne vlozi vsechny existujici soubory Figures/ZadaniXXX.jpg
% Pokud toto makro zapoznamkujeme sazi se stranka s upozornenim
%\ThesisAssignmentImagePath{Figures/Zadani}

% Zadame soubor s digitalizovanou podobou prohlaseni
% Pokud toto makro zapoznamkujeme sazi se cisty text prohlaseni
\DeclarationImageFile{Figures/Prohlaseni.jpg}



% Zacatek dokumentu
\begin{document}

% Nechame vysazet titulni strany.
\MakeTitlePages

% Asi urcite budeme potrebovat obsah prace.
\tableofcontents
\cleardoublepage	% odstrankujeme, u jednostranneho dokumentu o jednu stranku, u oboustrenneho o dve

% Jsou v praci tabulky? Pokud ano vysazime jejich seznam.
% Pokud ne smazeme nasledujici makro.
\listoftables
\cleardoublepage	% odstrankujeme, u jednostranneho dokumentu o jednu stranku, u oboustrenneho o dve

% Jsou v praci obrazky? Pokud ano vysazime jejich seznam.
\listoffigures
\cleardoublepage	% odstrankujeme, u jednostranneho dokumentu o jednu stranku, u oboustrenneho o dve


% Jsou v praci vypisy programu? Pokud ano vysazime jejich seznam.
\lstlistoflistings
\cleardoublepage	% odstrankujeme, u jednostranneho dokumentu o jednu stranku, u oboustrenneho o dve



% Zacneme uvodem
\section{Úvod}
\label{sec:Uvod}
Tento text je ukázkou sazby diplomové práce v \LaTeX{}u pomocí třídy dokumentů \verb|diploma|.
Pochopitelně text není skutečnou diplomovou prací, ale jen ukázkou použití
implementovaných maker v praxi. V kapitole \ref{sec:Typo} jsou ukázky použití
různých maker a prostředí. V kapitole \ref{sec:Conclusion} bude \uv{jako závěr}. Zároveň
tato kapitola slouží jako ukázka generování křížových odkazů v \LaTeX{}u.



\section{Ukázky sazby}
\label{sec:Typo}


\subsection{Ukázka nadpisů}
Toto je nadpis podsekce, generováno makrem \verb|\subsection|.


\subsubsection{subsection}

\paragraph{paragraph}


\subparagraph{subparagraph}
Ale tak hluboko se asi stejně nikdo nedostane.



\subsection{Sazba definic, vět atd.}
Určitě se bude hodit prostředí pro sazbu definice jako je definice binárního vyhledávacího
stromu, viz definice \ref{def:BinVyhlStrom}.

\begin{remark}
Následující definice a věty nedávají dohromady příliš smysl. Jsou tu jen pro ukázku.
\end{remark}


\begin{definition}
\label{def:BinVyhlStrom}
Binární strom je struktura definovaná nad konečnou množinou uzlů, která:
\begin{itemize}
\item neobsahuje žádný uzel,
\item je složena ze tří disjunktních množin uzlů:
      kořene, binárního stromu zvaného levý podstrom a~binárního stromu tzv.\ pravého podstromu.
\end{itemize}
\end{definition}

Pak by se taky mohla hodit nějaká věta a k ní důkaz.

\begin{theorem}
\label{VetaNeuspechSepar}
Průměrná časová složitost neúspěšného vyhledání  v~hashovací tabulce se  separátním zřetězením je $\Theta(1+\alpha)$, za předpokladu jednoduchého uniformního hashování.
\end{theorem}
\begin{proof}
Za předpokladu jednoduchého uniformního hashování se každý  klíč
$k$ hashuje se stejnou pravděpodobností do libovolného z~$m$ slotů
tabulky. Průměrný čas neúspěšného hledání klíče $k$ je proto
průměrný čas prohledání jednoho z~$m$ seznamů. Průměrná délka
každého takového seznamu je rovna faktoru naplnění $\alpha=n/m$.
Tudíž lze očekávat, že budeme nuceni prozkoumat $\alpha$ prvků.
Z~toho plyne, že celkový čas pro neúspěšné hledání (plus navíc
konstantní čas pro výpočet $h(k)$) je $\Theta(1+\alpha)$.
\end{proof}


\begin{example}
\label{priklad}
Mějme napsat funkci, která spočítá uzly ve stromu. Předpokládejme,
že binární strom je definován způsobem uvedeným v~definici
\ref{def:BinVyhlStrom} na straně \pageref{def:BinVyhlStrom}. Naše úloha se výrazně zjednoduší
uvědomíme-li si její rekurzivní charakter a~předpokládáme, že aktuální uzel je $R$.
\begin{itemize}
\item Je-li $R$ prázdný strom (tj.\ $R=NULL$), pak počet jeho uzlů je pochopitelně nula. Tím máme problém vyřešen.

\item V~opačném případě víme, že ve stromu  určitě jeden uzel existuje ($R$) a~počty uzlů
v~levém a~pravém podstromu se dají určit obdobným způsobem
rekurzivně. To znamená, že počet uzlů ve stromu s~kořenem $R$ je
$1 + \mathrm{pocet\_uzlu}(A) + \mathrm{pocet\_uzlu}(B)$
\end{itemize}

Počty uzlů pro jednotlivé podstromy se předávají jako výsledky
volání funkcí prostřednictvím zásobníku programu, nejsou tudíž
potřeba žádné pomocné proměnné.
\end{example}


\begin{remark}
Program z příkladu \ref{priklad} pochopitelně chybí,
ale můžete se podívat třeba na program uvedený ve výpisu \ref{src:Java}.
\end{remark}




\subsection{Výpisy programů}
Tato diplomová práce má nastaven výchozí jazyk Java, jak je vidět z výpisu~\ref{src:Java}. Výpis kódu \ref{src:Java} zároveň demonstruje možnost přímého vložení zdrojového kódu programu do textu práce. Druhou možností je načtení zdrojového kódu programu z externího souboru, viz výpis \ref{src:JavaExternal}. Pokud potřebujeme změnit programovací jazyk pro konkrétní výpis kódu, můžeme jeho to provést přímo v záhlaví prostředí \verb|lstlisting|. Výpis \ref{src:Pascal} je v jazyku Pascal. Všimněte si zvýraznění klíčových slov. 

\begin{remark}
Pro správnou sazbu je třeba pro odsazování používat tabulátory, nikoliv mezery.
\end{remark}


\begin{lstlisting}[label=src:Java,caption=Program v jazyce Java]
public class MyClass
{
	public int MyMethod(int a, int b)
	{
		while (a != b)
		{
			if (a < b)
				b -= a;
			else
				a -= b;
		}
	}
}
\end{lstlisting}


\lstinputlisting[label=src:JavaExternal,caption={Program v jazyce Java, načtený z externího souboru}]{MyClass.java}


\begin{lstlisting}[language=Pascal,label=src:Pascal,caption=Program v Pascalu]
procedure X(i : integer; var x : real);
begin
	x := i + 3;
end;
\end{lstlisting}



\subsection{Obrázky a tabulky}
A ještě si můžeme zkusit vysázet obrázek. Obrázek \ref{fig:SampleFigAbs} má určenu absolutní velikost,
zatímco obrázek \ref{fig:SampleFigRel} je určen relativně vůči šířce textu.


\InsertFigure{Figures/Obr1}{40mm}{Pokusný obrázek -- absolutní velikost}{fig:SampleFigAbs}


\InsertFigure{Figures/Obr1}{0.7\textwidth}{Pokusný obrázek -- relativní velikost}{fig:SampleFigRel}

\InsertSidewaysFigure{Figures/Obr1}{0.6\textheight}{Pokusný obrázek -- otočený naležato}{fig:SampleFigSideway}


A ještě zkusíme vysázet několik tabulek, ale jen kvůli seznamu tabulek v úvodu. Tabulka \ref{tab:ExpTable} představuje jednoduchou tabulku, která se svou šířkou pohodlně vejde do šířky textu. Velké tabulky, stejně jako obrázky, můžeme vysázet naležato. Ukázkou velké, komplikované tabulky\footnote{Pokud, ale píšete práci česky, měly by být tabulky také česky -- mě se jen nechtěla předělávat do češtiny.} je tabulka \ref{tab:ExpFilesDetailStats}.
 
\begin{table}
  \centering
  \begin{tabular}{|c|c|c|}
    \hline
    q & $\delta(q, 0)$ & $\delta(q, 1)$ \\
    \hline
    $q_0$ & $q_1$ & $q_0$ \\
    \hline
    $q_1$ & $q_1$ & $q_2$ \\
    \hline
    $q_2$ & $q_1$ & $q_0$ \\
    \hline
  \end{tabular}
  \caption{Pokusná tabulka}
  \label{tab:ExpTable}
\end{table}



\begin{sidewaystable}
\centering
\begin{tabular}{|l|rr|rr|rr|rr|}
\hline
File &
\multicolumn{2}{c|}{bible.txt} &
\multicolumn{2}{c|}{world.txt} &
\multicolumn{2}{c|}{law.txt} &
\multicolumn{2}{c|}{latimes.txt} \\
\hline
Language &
\multicolumn{2}{c|}{English} &
\multicolumn{2}{c|}{English} &
\multicolumn{2}{c|}{Czech} &
\multicolumn{2}{c|}{English} \\
Format &
\multicolumn{2}{c|}{Plain text} &
\multicolumn{2}{c|}{Plain text} &
\multicolumn{2}{c|}{Plain text} &
\multicolumn{2}{c|}{SGML} \\
\hline
Size of file [bytes] & 4047392 &  & 2473400 &  & 64573143 &  & 498360166 & \\
\hline
{\small{}Number of tokens} & 1532262 & 100\% & 684767 & 100\% & 19432898 & 100\% & 161254928 & 100\%\\
{\small{}Number of words} & 766131 & 50\% & 342383 & 50\% & 9716449 & 50\% & 70766067 & 43.885\%\\
{\small{}Number of nonwords} & 766131 & 50\% & 342384 & 50\% & 9716449 & 50\% & 80619289 & 49.995\%\\
{\small{}Number of controls} &  &  &  &  &  &  & 9869572 & 6.12\%\\
\hline
{\small{}Number of unique tokens} & 13791 & 100\% & 23564 & 100\% & 250570 & 100\% & 529482 & 100\%\\
{\small{}Number of unique words} & 13744 & 99.659\% & 23082 & 97.955\% & 246266 & 98.282\% & 524280 & 99.018\%\\
{\small{}Number of unique nonwords} & 47 & 0.341\% & 482 & 2.045\% & 4304 & 1.718\% & 3079 & 0.582\%\\
{\small{}Number of unique controls} &  &  &  &  &  &  & 2123 & 0.401\%\\
\hline
{\small{}Word average frequency} & 55.743 &  & 14.833 &  & 39.455 &  & 134.978 & \\
{\small{}Nonword average frequency} & 16300.66 &  & 710.34 &  & 2257.539 &  & 26183.595 & \\
{\small{}Control average frequency} &  &  &  &  &  &  & 4648.88 & \\
\hline
{\small{}Minimal length of word} & 1 &  & 1 &  & 1 &  & 1 & \\
{\small{}Maximal length of word} & 18 &  & 27 &  & 41 &  & 58 & \\
{\small{}Minimal length of nonword} & 1 &  & 1 &  & 1 &  & 1 & \\
{\small{}Maximal length of nonword} & 4 &  & 56 &  & 700 &  & 253 & \\
{\small{}Minimal length of control} &  &  &  &  &  &  & 3 & \\
{\small{}Maximal length of control} &  &  &  &  &  &  & 132 & \\
\hline
\end{tabular}
\caption{Experimental Files --- Detailed Statistics}
\label{tab:ExpFilesDetailStats}
\end{sidewaystable}


\section{Závěr}
\label{sec:Conclusion}
Tak doufám, že Vám tato ukázka k něčemu byla. Další informace najdete v~publikacích
\cite{goossens94,lamport94}.

\bigskip
\begin{flushright}
Jiří Dvorský
\end{flushright}




\begin{thebibliography}{99}

\bibitem{goossens94} Goossens, Michel,
\textit{The \LaTeX\ companion,} New York: Addison, 1994.

\bibitem{lamport94} Lamport, Leslie,
\textit{\LaTeX: a document preparation system: user's guide and reference manual},
New York: Addison-Wesley Pub. Co., 1994.

\end{thebibliography}


\appendix
\section{Grafy a měření}
Tohle je příloha k práci. Většinou se sem dávají grafy, tabulky, které by vzhledem
ke svému počtu překážely v textu diplomky.
\clearpage

\InsertFigure{Figures/Graf}{0.7\textwidth}{Nějaký graf}{fig:SampleGraph}

\end{document}
