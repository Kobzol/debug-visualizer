\documentclass[bc,male,python,dept460]{diploma}						% jednostranny dokument
%\documentclass[bc,male,python,dept460,twoside]{diploma}		% oboustranny dokument

\usepackage[english, czech]{babel}
\usepackage[T1]{fontenc}
\usepackage[
	backend=biber,
	autolang=other,
	style=iso-numeric,
	sorting=none,
	sortlocale=cs_CZ,
  bibencoding=UTF8]{biblatex}
\usepackage[bottom]{footmisc}

\renewcommand\multinamedelim{\addcomma\space}	%  redefinice oddělování jmén autorů dle ISO690:2011
\renewcommand\finalnamedelim{\addspace\mainsstring{and}\space}

\addbibresource{references.bib}

% remark - poznámka
% definition - definice
% theorem - věta
% example - příklad
% \begin{lstlisting}[label=src:Java,caption=Program v jazyce Java]
% \lstinputlisting[label=src:JavaExternal,caption={Program v jazyce Java, načtený z externího souboru}]{MyClass.java}
% \InsertFigure{Figures/Obr1}{40mm}{Pokusný obrázek -- absolutní velikost}{fig:SampleFigAbs}
% \InsertFigure{Figures/Obr1}{0.7\textwidth}{Pokusný obrázek -- relativní velikost}{fig:SampleFigRel}
% \InsertSidewaysFigure{Figures/Obr1}{0.6\textheight}{Pokusný obrázek -- otočený naležato}{fig:SampleFigSideway}

\ThesisAuthor{Jakub Beránek}
\ThesisTitle{Vizualizace ladění aplikací}
\EnglishThesisTitle{Visualization of application debugging}

\SubmissionDate{29. dubna 2016}

\AccessRestriction{Zde vložte text dohodnutého omezení přístupu k Vaší práci, chránící například firemní know-how.
Zde vložte text dohodnutého omezení přístupu k Vaší práce, chránící například firemní know-how.
A zavazujete se, že\par
\begin{enumerate}
\item o práci nikomu neřeknete,
\item po obhajobě na ni zapomenete a
\item budete popírat její existenci.
\end{enumerate}
A ještě jeden důležitý odstavec. A ještě jeden důležitý odstavec.
A ještě jeden důležitý odstavec. A ještě jeden důležitý odstavec.
A ještě jeden důležitý odstavec. A ještě jeden důležitý odstavec.
Konec textu dohodnutého omezení přístupu k Vaší práci.}

\Thanks{Rád bych na tomto místě poděkoval všem, kteří mi s prací pomohli, protože bez nich by tato práce nevznikla.}

\CzechAbstract{Tato bakalářská práce se zabývá vizualizací ladění programů napsaných v jazyce C a C++. První část pojednává o
obecných principech ladění programů. %Jsou zde popsány konstrukce, které se při ladění používají, způsob, jakým ladící nástroje
%provádějí ladění programů a také existující ladících nástrojy a jejich grafické nástavby.
Druhá část popisuje běžně používané ladící nástroje a jejich grafické nádstavby.
Třetí část se zabývá implementací grafického nástroje, který vizualizuje pamět a stav procesu během jeho ladění za využití existujících ladících nástrojů  }

\CzechKeywords{ladění programů, vizualizace paměti}

\EnglishAbstract{This is English abstract. This is English abstract. This is English abstract. This is English abstract. This is English abstract. This is English abstract.}

\EnglishKeywords{typography, \LaTeX, master thesis}

\AddAcronym{API}{Application Programmable Interface}
\AddAcronym{GDB}{The GNU Project Debugger}
\AddAcronym{GNU}{GNU's Not Unix!}
\AddAcronym{GUI}{Graphical user interface}
\AddAcronym{IPC}{Inter-process communication}

% Zadame cestu a jmeno souboru ci nekolika souboru s digitalizovanou podobou zadani prace
% Pri sazbe se pak hledaji soubory Figures/Zadani1.jpg, Figures/Zadani2.jpg atd.
% Do diplomove prace se postupne vlozi vsechny existujici soubory Figures/ZadaniXXX.jpg
% Pokud toto makro zapoznamkujeme sazi se stranka s upozornenim
%\ThesisAssignmentImagePath{Figures/Zadani}

% Zadame soubor s digitalizovanou podobou prohlaseni
% Pokud toto makro zapoznamkujeme sazi se cisty text prohlaseni
\DeclarationImageFile{Figures/Prohlaseni.jpg}

% Zacatek dokumentu
\begin{document}

% Nechame vysazet titulni strany.
\MakeTitlePages

% Asi urcite budeme potrebovat obsah prace.
\tableofcontents
\cleardoublepage	% odstrankujeme, u jednostranneho dokumentu o jednu stranku, u oboustrenneho o dve

% Jsou v praci tabulky? Pokud ano vysazime jejich seznam.
\listoftables
\cleardoublepage	% odstrankujeme, u jednostranneho dokumentu o jednu stranku, u oboustrenneho o dve

% Jsou v praci obrazky? Pokud ano vysazime jejich seznam.
\listoffigures
\cleardoublepage	% odstrankujeme, u jednostranneho dokumentu o jednu stranku, u oboustrenneho o dve

% Jsou v praci vypisy programu? Pokud ano vysazime jejich seznam.
\lstlistoflistings
\cleardoublepage	% odstrankujeme, u jednostranneho dokumentu o jednu stranku, u oboustrenneho o dve

% Zacneme uvodem
\section{Úvod}
	Ladění je nezbytná součást vývoje programů, která dovoluje programátorům detailně sledovat a ovládat laděný proces, aby v něm mohli odhalit
	chyby a lépe pochopit jeho průběh. K tomuto slouží ladící programy, které vytváří asociaci mezi zdrojovým kódem a binárním spustitelným
	souborem a poskytují tak tvůrci kódu možnost ladit kód na vysoké úrovni abstrakce, tj. na úrovni samotného zdrojového kódu. Cílem této práce
	je vývoj grafického rozhraní, které bude nezávislé na použitém ladícím programu a bude umožňovat vizualizaci laděného programu pro lepší
	pochopení jeho vnitřního stavu.
	
	\par První kapitola popisuje obecné principy ladění, které jsou společné pro všechny ladící programy. Druhá kapitola pojednává o existujících
	ladících programech a jejich grafických rozhraních. Třetí kapitola se věnuje návrhu a vývoji rozhraní pro přístup k ladícím programům a grafického
	rozhraní pro vizualizaci ladění programů.
	
	\par Tato práce je zaměřená na programy pro operační systémy založené na Linuxovém jádře. Pojmem Linuxový systém se v této práci myslí
	libovolná distribuce Linuxu. Pro označení programů, které umožňují ladění jiných programů, je v této práci používán termín debugger,
	jelikož se jedná o často používaný programátorský termín a v češtině pro něj neexistuje zavedená alternativa.
	
\section{Principy ladění programů}
	Tato kapitola popisuje obecné principy fungování ladících nástrojů, způsob mapování binárních instrukcí programu zpět do jeho zdrojového kódu,
	krokování běžícího procesu a běžné konstrukce používané při ladění. Konkrétně je popis zaměřen na programy napsané v jazycích C a C++
	v prostředí Linuxových systémů používající procesory z rodiny Intel x86. %TODO: důvod?%
	Popsané principy jsou ale obecné a lze je aplikovat na libovolný operační systém.
		
	\subsection{Signály}
		Pro ladění programu je nutné mít možnost číst jeho paměť, aby šly zkoumat hodnoty proměnných za jeho běhu, a také ho zastavit, jelikož
		programy za běhu provádějí obrovské množství instrukcí za vteřinu a zkoumat takto rychle se měnící datový tok by bylo obtížné.
		Aby šlo proces zastavit, musí mu jiný proces anebo sám operační systém zaslat signál.
		Signály jsou zprávy, které lze zaslat běžícímu procesu, ten si je může odchytit a zareagovat na ně.\cite[21]{tanenbaum}
		Slouží pro meziprocesní komunikaci a fungují jako softwarová obdoba hardwarových přerušení procesoru.
		Jakmile proces obdrží signál, který očekává, tak si uloží hodnoty svých registrů a přejde do procedury, která tento signál obslouží.
		Pokud proces obdrží signál, pro který si nepřipravil žádnou reakci, tak se provede implicitně nadefinovaná akce pro daný typ signálu.
		V Linuxových systémech je definováno několik desítek standardních signálů, v závislosti na verzi a typu operačního systému.
		Na signály SIGKILL, sloužící k okamžitému ukončení procesu a SIGSTOP, sloužící k zastavení procesu, nemá proces možnost zareagovat ani
		zjistit, že mu byly poslány.
	
	\subsection{Krokování}
		Operační systémy obvykle poskytují nástroj, pomocí kterého lze buď spustit proces, anebo se připojit k již běžícímu procesu, a následně ho ovládat
		a přistupovat k jeho paměti. Linuxové systémy pro tento účel poskytují systémové volání \textbf{ptrace}\footnote{http://linux.die.net/man/2/ptrace},
		které umožňuje zachytávat signály zaslané sledovanému procesu. Proces sledovaný pomocí funkce ptrace je zastaven při přijetí jakéhokoliv signálu
		(kromě signálu SIGKILL, který se pokusí proces okamžitě ukončit). Tohoto mechanismu využívají ladící nástroje, které proces sledovaný pomocí
		ptrace můžou po jeho zastavení znovu spustit, přistupovat k jeho paměti a ovlivňovat jeho průběh. Pokud je sledovaný proces potomkem procesu,
		který ho sleduje, bude při jeho spuštění vyvolán signál SIGTRAP, který dovolí rodičovi odchytit začátek provádění potomka. Jakmile je proces
		zastavený, může ho ladící nástroj tzv. krokovat, tedy spouštět instrukci po instrukci. K tomu lze použít funkci ptrace s příznakem PTRACE\_SINGLESTEP,
		která provede přesně jednu instrukci v laděném procesu (proces se také zastaví, pokud se dostane na vstupní nebo výstupní bod systémového volání).
		Ladící nástroje obvykle nabízí krokování na vyšší úrovni než pouze po jedné instrukci, jelikož to by bylo zbytečně zdlouhavé (u vyšších programovacích
		jazyků se jeden řádek zdrojového kódu může mapovat na desítky až stovky instrukcí). Obvykle jsou dostupné následující krokovací akce:
		\begin{description}
			\item[Krok po řádku] - program se obnoví, provede instrukce odpovídající jednomu řádku zdrojového kódu a poté se opět zastaví
			\item[Krok dovnitř funkce] - funguje stejně jako krok po řádku, ale program se zastaví i při zavolání funkce
			\item[Krok ven z funkce] - program bude pokračovat, dokud neskončí funkce, ve které se právě nachází
		\end{description}
		
	\subsection{Obousměrné mapování zdrojového kódu na instrukce}
		Aby mohly ladící nástroje nabízet krokování na úrovni (řádků) zdrojového kódu, musí umět namapovat zdrojový kód na instrukce vygenerovaného
		spustitelného programu i instrukce zpět na zdrojový kód. Jelikož programy psané v jazycích C a C++ jsou kompilované a po jejich překladu nejsou
		ve výsledném binárním souboru téměř žádné informace o jejich zdrojovém kódu, musí být přeloženy ve speciálním režimu, který při překladu vygeneruje
		metadata s mapováním zdrojového kódu a vloží je do přeloženého programu. V překladačích jazyka C/C++ se tohoto dá standardně dosáhnout použitím
		řepínače \textbf{-g}. Existuje několik formátů ukládání těchto metadat, dnešním de facto standardem na Linuxových systémech je DWARF
		\footnote{http://dwarfstd.org}. Ten ukládá proměnné, datové typy, procedury a další údaje ze zdrojového kódu ve stromové struktuře.
		Pro ušetření místa obsahuje instrukce pro speciální konečný automat, který implementují ladící nástroje a pomocí něho poté získávají
		informace o původním zdrojovém kódu.
		
		\vspace{5mm}
		
		\par Samotné mapování zdrojového kódu není pro ladící nástroj užitečné, pokud je výsledný program zoptimalizovaný překladačem. Po optimalizaci
		totiž program nemusí obsahovat všechny původní proměnné, funkce a jeho průběh ani nemusí přesně odpovídat jeho zdrojovému kódu. Při použité málo agresivní
		optimalizace někdy lze programy úspěšně ladit, ale pro zajištění co nejpřesnějšího ladění programů je obvykle nutné optimalizace úplně vypnout. Toho lze
		v překladačích obvykle dosáhnout použitím přepínače \textbf{-O0}.
	
	\subsection{Běžné konstrukce ladících nástrojů}
		\begin{description}
			\item[Breakpoint]
				Většina ladících nástrojů poskytuje svým uživatelům možnost zastavit běh laděného procesu pomocí tzv. breakpointu.
				Jedná se o označení řádku v zdrojovém kódu programu, na kterém se program za běhu zastaví a umožní tak uživateli prozkoumat paměť procesu a krokovat ho.
				Nejčastěji je implementován tak, že ladící nástroj nejdříve zjistí z daného řádku adresu instrukce ve vygenerovaném spustitelném souboru, kterou tento
				řádek představuje, uloží si ji a nahradí ji instrukcí přerušení s kódem 3. Toto přerušení je určeno speciálně pro ladění procesů, jelikož generuje
				instrukci o velikosti jednoho bytu, a lze jím tak nahradit libovolnou instrukci\cite[306]{intel}. Pokud by měla více než jeden byte, mohlo by ses
				stát, že by tato instrukce přepsala více než jednu instrukci, což by mohlo způsobit nevalidní chování programu. Jakmile program během svého běhu
				provede tuto instrukci, vyvolá se signál SIGTRAP, který ladící nástroj odchytí a laděný proces se tímto zastaví. Pokud se uživatel rozhodne proces
				opět sputit, ladící nástroj zkopíruje původní instrukci programu (kterou si dříve uložil) na místo, kde vložil přerušení, nastaví na ni ukazatel
				příští instrukce a proces opět spustí.
				Některé procesory nabízí také hardwarový breakpoint, který sice může být rychlejší, ale obvykle kvůli tomu, že je implementován v hardwaru, tak poskytuje
				vytvoření pouze několika breakpointů zároveň.
			\item[Tracepoint]
				V některých případech není možné laděný proces pozastavit k prozkoumání jeho paměti, jelikož jeho průběh může záviset na reálně uběhlém čase a zastavení
				tedy může způsobit, že program neproběhne korektně. Pro tyto situace lze použít tracepoint, u kterého se uvede lokace v programu a paměť, která má být
				sledována. Pokaždé, když se laděný proces dostane na tuto lokaci, tak je uložena sledovaná paměť a po skončení běhu procesu si lze zpětně prohlédnout,
				jak se tato paměť v průběhu programu měnila.
			\item[Watchpoint]
				Pokud je potřeba zastavit program ne na konkrétním místě, ale při změně dané hodnoty v paměti, lze použít watchpoint. Ten se může hodit například pro
				kontrolu změn globálních proměnných. Pokud nenabízí procesor hardwarovou podporu pro watchpointy, ladící nástroj prochází laděný proces instrukci po
				instrukci, testuje hodnotu sledované paměti a pokud se tato hodnota změní, tak program zastaví. Tento proces může zpomalit laděný proces až o
				dva řády\footnote{https://sourceware.org/gdb/onlinedocs/gdb/Set-Watchpoints.html}.
			\item[Catchpoint]
				Tuto konstrukci lze použít pro zachycení událostí procesu, jako jsou načtení sdílené knihovny, vyvolání hardwarové či softwarové výjimky, provedení
				systémového volání anebo přijetí signálu. Většina těchto událostí je ze své podstaty asynchronní, nelze u nich tedy dopředu určit, kdy budou zavolány
				a použití breakpointu tedy není možné.
		\end{description}
		
\section {Existující ladící nástroje}
	Ladících nástrojů pro jazyky C a C++ existuje velké množství, v této kapitole jsou popsány dva z nejpoužívanějších nástrojů, GDB a LLDB, které byly dále
	použity při implementaci vizualizačního nástroje. Jsou zde taky popsány jejich vybrané grafické nádstavby.

	\subsection{GDB}
		GDB (The GNU Debugger) je standardním ladícím nástrojem pro Linuxové systémy, často je v těchto systémech už předinstalovaný.
		Jeho hlavním zaměřením je ladění programů napsaných v jazycích C a C++, ale podporuje mimo jiné i Adu, Objective-C, Pascal, Fortran, Javu
		a Go\cite{gdb-languages}. Podporuje velké množství rodin procesorů, například ARM, AVR, Itanium, MIPS, PowerPC, SPARC a samozřejmě x86 i x86-64.
		Lze jej spustit i na platformě Windows pomocí prostředí emulujících Linux, jako je Cygwin nebo MinGW. Byl vydán v roce 1986 a k roku
		2016 stále vycházejí nové verze.
		
		\par Umí spolupracovat s programy přeloženými libovolným překladačem jazyků C a C++, pokud je dodržen jejich standard. Navíc ještě obsahuje
		speciální podporu pro překladač GCC (GNU Compiler Collection), který pro něj umí vygenerovat dodatečná ladící metadata.
		GDB obsahuje základní funkce nezbytné pro každý ladící nástroj, jako je načtení ladících metadat v mnoha formátech (podporuje DWARF i několik dalších
		formátů), vytváření breakpointů, tracepointů a watchpointů, krokování programu a čtení i zápis paměti programu.
		Mimo to ale nabízí i pokročilé funkce, které ovšem nemusí být podporované všemi procesory a platformami, s kterými GDB jinak umí pracovat.
		
		\begin{description}
			\item[Vzdálené ladění]
				GDB dokáže být spuštěn na jednom zařízení a ladit program spuštěný na jiném zařízení pomocí síťové komunikace (obvykle pomocí protokolu TCP).
				Toto může být užitečné, pokud není dostupný fyzický přístup k systému, který je potřeba odladit.
				Vzdálené ladění se dá využít také k ladění jádra (kernelů) operačního systému, čehož je využito například v programu KGDB, který se používá k ladění
				jader operačních systémů Linux a FreeBSD pomocí sériového připojení.
			\item[Ladění vícevláknových aplikací]
				Pokud GDB ladí program, který využívá více než jedno vlákno, může pracovat v několika rozlišných módech\cite{gdb-multithreading}.
				V All-stop módu se při zastavení jednoho vlákna zastaví také všechna ostatní vlákna, aby šlo mezi vlákny přepínat a číst jejich paměť bez toho,
				aby se paměť mezitím nějak změnila. Pokud je nutné zastavit pouze jedno vlákno, tak, aby ostatní pokračovala v běhu, lze použít tzv. Non-stop mód,
				který vždy zastaví pouze vlákno, které narazí na breakpoint, a zbytek vláken nechá běžet. S tímto módem je vhodné použít asynchronní ovládání GDB,
				pomocí kterého lze zasílat ladící příkazy programu i za jeho běhu a ovládat tak pouze zastavené vlákno, i když zbytek vláken stále běží.
			\item[Provádění výrazů]
				Pomocí GDB lze nejenom číst a zapisovat paměť laděného procesu na úrovni bytů v adresním prostoru procesu, ale v podporovaných jazycích, hlavně
				v C a C++, lze také provádět libovolné jazykové výrazy, volat funkce programu a systémová volání a pracovat s hodnotami na úrovni proměnných
				laděného programu.
			\item[Spolupráce s Valgrindem]
				Valgrind je nástroj pro profilování a kontrolu paměťové korektnosti programů, který se využívá k hledání paměťových chyb, jako je například memory
				leak. Vytváří virtuální stroj, ve kterém spouští zkoumaný program a kvůli této vlastnosti jej nelze ladit klasickými přístupy. GDB poskytuje
				možnost připojit se k programu spuštěnému ve Valgrindu a vzdáleně ho takto ladit.
			\item[Analýza logu z ukončeného procesu]
				Procesy, které se ukončí s chybou, např. po vyvolání výjimky, můžou vygenerovat výpis paměti (core dump), který lze poté načíst v GDB a zanalyzovat ho.
				Lze tak například zobrazit stav zásobníku volání funkcí v momentu, kdy program zhavaroval, a zjistit tak, který kód programu způsobuje chybu.
			\item[Zpětné provádění instrukcí]
				Při ladění nastává často situace, kdy proces zajde moc daleko a přeskočí instrukci, kterou chce uživatel zkoumat. GDB umí spouštět určité instrukce
				zpětně, a může tedy krokovat program nejenom dopředu, ale i dozadu. Všechny změny a vedlejší efekty, které proběhly v paměti, jsou tak smazány a
				navráceny do původního stavu (pokud to daná platforma a stav programu dovoluje).
		\end{description}
		
		GDB nemá vlastní grafické rozhraní, je ovládán z příkazové řádky. Kromě toho ale podporuje také spouštění skriptů v Pythonu pomocí API,
		které bylo použito pro implementaci vizualizačního nástroje a je popsáno dále v textu.
	\subsection{LLDB}
		Ladící nástroj LLDB\footnote{http://llvm.org} je založen na sadě knihoven, které využívají infrastruktury LLVM a překladače Clang.
		LLVM je univerzální překladače, který dokáže překládat velké množství jazyků do své vnitřní, jazykově nezávislé reprezentace, kterou umí
		optimalizovat a vygenerovat z ní dále spustitelný soubor pro libovolnou kompatibilní platformu. Obsahuje také kompletní implementaci standardní
		knihovny jazyka C++, která plně podporuje jeho nejnovější standard, C++11. Clang je nádstavbou LLVM, která analyzuje a překládá programy v jazyce
		C a C++. Celá LLVM architektura je postavena na modulárních komponentech, které spolupracují a dají se lehce využít ve formě knihovny.
		Nabízí tak modernější alternativu k programům GCC a GDB. Ty jsou sice stabilnější a prověřenější, ale jelikož existují už desítky let a musí udržovat
		zpětnou kompatibilitu, tak je těžší je využít jako modul do jiného programu. Z tohoto důvodu rovněž pomaleji přecházejí k novým standardům.
		
		\vspace{5mm}
		
		\par Umí ladit programy napsané v jazycích C, C++, Objective-C a Swift na platformách OS X, Linux, Free BSD a Window. Podporuje tedy méně jazyků i platforem,
		než GDB, ale narozdíl od něho je podporován, a stal se také standardním ladícím nástrojem, i operačními systémy OS X a iOS. Nabízí většinu standardních
		funkcí ladících nástrojů, jako je krokování kódu, vytváření breakpointů a čtení a zápis paměti procesu. Jelikož je stále ve vývoji, tak zatím neobsahuje 
		některé pokročilejší funkce, které nabízí GDB, například zpětné provádění instrukcí.

	\subsection{Grafická rozhraní}
		Grafických rozhraní pro debuggery GDB a LLDB existuje několik desítek. Některé z nich jsou samostatné programy podporující pouze ladění, další jsou
		jednou z mnoha součástí integrovaných vývojových prostředí. Rozhraní těchto nástrojů se obvykle skládá z textového editoru, který obsahuje zdrojový kód,
		ovládacích prvků, které kontrolují průběh laděného procesu. Dále také často nabízí manipulaci a zobrazování registrů, lokálních proměnných a parametrů
		funkcí. Ukázky uživatelského rozhraní jednotlivých programů si lze prohlédnout v příloze \ref{appendix:gui}. Následuje popis jednoho zástupce
		ze skupiny samostatných (DDD), integrovaných (Clion) a textových (TUI) uživatelského rozhraní pro ladění programů.
	
		\begin{description}
			\item[TUI]
				Text User Interface je grafickým rozhraním vestavěným přímo v GDB, které zobrazuje stav průběhu v několik terminálových oknech pro větší přehlednost
				programu. Je postaveno na knihovně curses, která umožňuje vytvářet textové uživatelské rozhraní s pokročilými funkcemi přímo v terminálu. TUI lze
				spustit předáním parametru \textbf{-tui} při spouštění GDB anebo stisknutím kláves CTRL+X či spuštěním příkazu \emph{tui enable} za jeho běhu.
				LLDB obsahuje podobné rozhraní také, ale není zatím oficiální součástí nástroje, jedná se pouze o nezávazně vyvíjený doplněk.
			\item[DDD]
				DDD, neboli Data Display Debugger, je grafické prostředí podporující velké množství ladících nástrojů, mimo jiné GDB, pydb, DBX nebo Ladebug.
				Mimo klasického zobrazování zdrojového kódu programu nabízí i pokročilé vizualizační funkce. Umí kreslit grafy z hodnot paměti procesu
				anebo zobrazovat vztahy mezi objekty v paměti ve formě grafu. Jeho poslední verze vyšla v roce 2009, není už tedy v současnosti aktivně udržován.		
			\item[Clion]
				Clion je integrované vývojové prostředí založené na vývojové platformě IntelliJ. Nabízí mimo jiné statickou analýzu kódu psaného v jazycích
				C a C++, což pomáhá v odhalování velkého množství chyb již během psaní programu. Tato analýza zároveň usnadňuje ladění kódu poskytováním
				automatického doplňování výrazů a proměnných, které lze v laděném procesu sledovat. Během ladění Clion zobrazuje vedle názvů proměnných
				ve zdrojovém kódu jejich současnou hodnotu, což velmi urychluje pochopení stavu výpočtu.
		\end{description}

\section{Implementace vizualizačního nástroje}
	Jazyky C a C++ jsou velmi komplexní a dovolují programátorům pracovat s hardwarem počítače na značně nízké úrovni. Kvůli mnohým vlastnostem,
	jako je absence automatické správy paměti nebo možnost provádět potencionálně nebezpečné operace s paměťovými ukazateli, jsou velmi náchylné ke vzniku
	těžko odhalitelných chyb. Používání debuggeru, který dokáže detailně zkoumat paměť běžícího procesu a krokovat ho, je tedy v těchto jazycích nutností,
	bez které by byl vývoj C a C++ aplikací výrazně pomalejší.
	Existující debuggery, jako je GDB nebo LLDB, mají velké množství užitečných funkcí, ale samy o sobě poskytují svému uživateli pouze ladění
	pomocí terminálu. To je pro velké programy velmi pomalé a nepřehledné řešení. Nad těmito nástroji proto vznikla různá grafické rozšíření,
	ať už samostatná nebo integrovaná v komplexních vývojových prostředích. Tato uživatelská rozhraní umožňují mnohem pohodlnější a jednodušší ladění programů.
	Díky tomu může být ladění přirozenou součástí vývoje programů, a ne pouze obtížnou činností nutnou při hledání skrytých chyb.
	I když grafické uživatelské rozhraní ladění značně usnadňuje, tak reprezentace paměti se v něm většinou omezuje na pouhý textový výpis názvů proměnných
	a jejich hodnot. To postačuje pro hledání chyb v programu, ale nedovoluje to plně zobrazit vztahy mezi objekty a jejich umístění v paměti.
	Zobrazení grafické reprezentace objektů v paměti může být také užitečné pro pochopení a sledování průběhu určitých (např. třídících) algoritmů.

	\subsection{Specifikace požadavků}
	Cílem této práce bylo navrhnout a naimplementovat program, který bude ladění programu vizualizovat ve formě grafu objektů v paměti, a otestovat,
	jestli tato grafická reprezentace usnadňuje ladění programů a pochopení průběhu jednoduchých algoritmů.
	Program by také měl poskytovat přístup k běžným funkcím debuggerů.
	Následuje seznam základních funkcí, které by měl svým uživatelům nabízet.
	
	\begin{itemize}
		\item Asynchronní komunikace s debuggerem
		\item Načítání binárních i zdrojových souborů
		\item Zobrazování zdrojového i strojového kódu programu
		\item Vytváření a odebírání breakpointů
		\item Manipulace s procesem (spuštění, pozastavení, zastavení)
		\item Krokování (krok po řádku, krok dovnitř funkce, krok ven z funkce)
		\item Komunikace s procesem v reálném čase (přes standardní vstup a výstup)
		\item Přepínání zásobníkových rámců
		\item Přepínání vláken
		\item Zobrazování registrů a paměti na úrovni bytů
		\item Manipulace s hodnotami proměnných
		\item Vizualizace objektů v paměti ve formě grafu
	\end{itemize}
	
	S touto sadou funkcí by měl být tento program plně použitelný pro ladění reálně používaných aplikací.
	
	\subsection{Architektura}
	\par Pro implementaci programu jsem zvolil programovací jazyk \textbf{Python}\footnote{http://www.python.org} ve verzi 2.7. Novější verze 3.x nebyla použita
	z důvodu zachování kompatibility s již existujícími API pro debuggery, které v době tvorby této práce Python 3 ještě zcela nepodporovaly. Pro případnou
	budoucí konverzi do Pythonu 3 lze použít automatizované nástroje, například \textbf{2to3} \footnote{https://docs.python.org/2/library/2to3.html}.
	Python jsem vybral, protože je vhodný k rychlému vývoji aplikací, existuje pro něj několik rozhraní pracujících s debuggery a umí snadno používat
	kód napsaný v jazycích C a C++. Navíc je multiplatformní, což by usnadnilo případný port aplikace na jiný operační systém.
	
	\vspace{5mm}
	
	\par Pro vývoj grafického rozhraní programu jsem vybral knihovnu \textbf{GTK+ 3}\footnote{http://www.gtk.org}. Jedná se o volně dostupný
	\footnote{Pod licencí LGPL 2.1}, multiplatformní grafický software umožňující tvorbu uživatelského rozhraní, který je používán mimo jiné i
	některými manažery plochy pro Linux (např. v GNOME\footnote{https://www.gnome.org/technologies}). Kromě možnosti tvorby vlastních
	GUI prvků obsahuje několik desítek běžně používaných prvků, které jsou předpřipravené k okamžitému použití a usnadňují tak rychlý vývoj aplikací.
	
	\vspace{5mm}
	
	\par Vizualizační nástroj jsem pojmenoval a dále jej v textu budu označovat jako \textit{Devi}.
	Nástroj je tvořen dvěmi samostatnými komponentami, aplikací s grafickým rozhraním, která slouží k vizualizaci a k interakci s laděným procesem, a knihovny,
	která poskytuje rozhraní pro komunikaci s libovolným debuggerem. Obrázek \ref{fig:DeviArchitecture} zachycuje pohled na architekturu aplikace z vysoké úrovně.
	Uživatelské rozhraní komunikuje s knihovnou, která zprostředkovává komunikaci s debuggerem. Na obrázku jsou zobrazeny tři implementace této komunikační vrstvy,
	které jsou popsány dále v textu. Debugger zajišťuje komunikaci a manipulaci s laděným procesem. Celá aplikace je tak rozdělena do několika
	vrstev, které jsou na sobě nezávislé.
	
	\InsertFigure{Figures/bak_architektura}{\textwidth}{Architektura Devi}{fig:DeviArchitecture}
		
	\subsection{Knihovna pro komunikaci s debuggery}
		Tato knihovna tvoří rozhraní pro komunikaci s libovolným ladícím nástrojem. Definuje abstraktní typy popisující laděný proces a není tak závislá na
		použitém ladícím nástroji. Není nijak závislá ani na grafickém rozhraní nástroje, lze ji tedy použít pro programovatelné ovládání debuggerů i v jiném
		projektu. Knihovna obsahuje třídy představující části laděného procesu, které jsou dostatečně obecné na to, aby se daly aplikovat na libovolný
		proces i debugger. Následuje popis těchto typů.
		
		\begin{description}
			\item[Type] představuje datový typ proměnné, obsahuje jeho název, kategorii a velikost
			\item[Variable] představuje proměnnou, obsahuje její název, datový typ, adresu v paměti, hodnotu a potomky
			\item[Frame] představuje rámec zásobníku, obsahuje jeho úroveň v zásobníku a název a lokaci funkce, z které je vyvolán % TODO
			\item[InferiorThread] představuje vlákno procesu, obsahuje jeho identifikátor, název, stav a současný zásobníkový rámec
			\item[ThreadInfo] uchovává seznam všech vláken v procesu a také současně zvolené vlákno
			\item[Breakpoint] představuje breakpoint, obsahuje jeho identifikátor a zdrojový soubor a řádek, na kterém je umístěn
			\item[Register] představuje registr procesoru, obsahuje jeho název a hodnotu
		\end{description}
		
		\vspace{5mm}
		
		\par Dále jsou v knihovně třídy, které tvoří obecné API pro přístup k debuggeru. Slouží jako rodičovské třídy pro jednotlivé implementace
		komunikace s debuggerem. Jejich smyslem není implementovat společné chování, jelikož jednotlivé debuggery a implementace přístupu k nim
		se od sebe můžou značně lišit, a tak poskytují pouze rozhraní. Následuje seznam tříd tohoto rozhraní.
		
		\begin{description}
			\item[Debugger] představuje abstrakci ladícího nástroje, který umožňuje načtení binárního souboru programu, jeho spuštění, zastavení
			a krokování. Dále také uchovává stav laděného procesu a obsahuje všechny ostatní komponenty rozhraní vyjmenované níže.
			\item[VariableManager] - se stará o čtení registrů a paměti procesu a o získávání a změnu jeho proměnných
			\item[ThreadManager] se stará o získávání a volbu vláken a zásobníkových rámců
			\item[FileManager] se stará o získávání současné pozice a hlavního souboru laděného procesu. Poskytuje také převod zdrojového kódu
			na strojový kód.
			\item[BreakpointManager] se stará o přidávání, mazání a načítání breakpointů
			\item[IOManager] se stará o komunikaci s laděným procesem pomocí čtení z jeho standardního a chybového výstupu a zápisu do jeho standardního vstupu
		\end{description}
		
		Pro účely této práce jsem v této knihovně vytvořil a otestoval několik implementací pro komunikaci s GDB a LLDB.

		\subsubsection{Python API pro GDB}
		GDB obsahuje rozhraní \footnote{https://sourceware.org/gdb/onlinedocs/gdb/Python-API.html}, které nabízí možnost načtení skriptů v Pythonu 2, které
		můžou ovládat GDB a přistupovat k jeho vnitřním funkcím. Většina nejpoužívanějších funkcí GDB je tak dostupná ve formě tříd a funkcí.
		Funkce GDB, které v API nejsou obsaženy, lze vyvolat pomocí přímého provádění textových příkazů.
		Nevýhoda tohoto přístupu je, že toto API neumí samo spustit instanci GDB a musí tak být načteno v již běžícím GDB procesu.
		Nelze jej tedy použít přímo v kódu jiné aplikace bez použití IPC. Tento problém jsem vyřešil TCP komunikací se skriptem běžícím
		v procesu GDB. Jelikož je ale toto řešení komplikovanější než ostatní způsoby práce s GDB a při testování
		nebylo stabilní, tak jsem jeho implementaci nadále nerozšiřoval.
		
		\subsubsection{Python API pro LLDB}
		Stejně jako GDB, i LLDB poskytuje API v Pythonu. Lze jej načíst do procesu LLDB a automatizovat tak průběh ladění, stejně jako u dříve zmíněného GDB
		rozhraní. Navíc jej ale lze také použít jako knihovnu, která sama vytváří instanci LLDB a je tedy jednoduché ji použít přímo v externím kódu.
		Pomocí tohoto API jsem vytvořil komunikační vrstvu, která implementuje většinu potřebných funkcí pro tvorbu vizualizačního nástroje.
		Nicméně stejně jako celé LLDB je toto API zatím ve vývoji a obsahuje drobné chyby, například nepřesné zobrazování stavu vláken. Kvůli těmto nedostatkům
		jsem do této implementace nepřidával podporu ladění vícevláknových aplikací.
		
		\subsubsection{Protokol GDB/MI}
		MI je komunikační rozhraní pro ovládání GDB pomocí textových příkazů. Nejedná se o klasické příkazy používané při manuálním ovládání GDB, ale o
		speciální příkazy přizpůsobené pro jednoduché strojové zpracování. Obdobně jako Python API pro GDB, obsahuje rozhraní pro nejpoužívanější funkce GDB a
		zbytek funkcí lze používat pomocí běžných textových příkazů. Použití MI je běžným a doporučeným \cite{gdb-mi-usage} způsobem pro programovou komunikaci
		s GDB a tvorbu grafických rozhraní. Existují volně dostupné knihovny pro Python i C/C++, které umí komunikovat s GDB pomocí MI protokolu, nicméně za
		účelem pochopení fungování tohoto protokolu a možnosti přizpůsobení komunikace jsem vytvořil vlastní komunikační modul, který je popsán dále v textu.
		
		\begin{description}
			\item[Komunikace s debuggerem]
			\item[Komunikace s laděným procesem]
			MI implicitně přesměrovává veškerý výstup z laděného procesu na svůj vlastní výstup. Toto chování není moc užitečné, jelikož se výstup a vstup laděného
			procesu mísí s vstupem a výstupem samotného ovládání GDB. Pro vyřešení tohoto problému lze v GDB pomocí příkazu \textit{-inferior-tty-set}
			\footnote{https://sourceware.org/gdb/onlinedocs/gdb/GDB\_002fMI-Miscellaneous-Commands.html} nastavit terminál, s kterým bude laděný proces komunikovat.
			Při použití tohoto řešení se mi nepovedlo komunikovat s procesem v reálném čase kvůli vnitřnímu bufferování terminálu a problému s čtením z terminálu z
			jiného vlákna. % TODO: dát pryč předchozí větu?
			Rozhodl jsem se tedy použít třetí řešení, které při komunikaci s laděným procesem naprosto obchází GDB.
			% TODO: vložit obrázek s nákresem I/O komunikace
			
			\par Před zapnutím laděného procesu se v systémové složce pro dočasná data vytvoří tři speciální soubory, jeden pro vstup a dva pro výstup
			(klasický a chybový).
			Tyto soubory jsou vytvořeny pomocí systémového volání \textit{mkfifo}\footnote{http://linux.die.net/man/3/mkfifo}, které vytváří pojmenovanou FIFO pípu
			% TODO: pípa?
			reprezentovanou souborem. Takto vytvořené soubory lze poté otevřít pro čtení nebo zápis. Narozdíl od klasických souborů ale musí být tato pípa otevřená na
			obou svých koncích zároveň. Pokud se tedy nějaký proces pokusí otevřít pípu pro čtení, bude zablokován, dokud se nějaký jiný proces nepokusí otevřít danou
			pípu pro čtení. Obsah pípy není uložen na souborovém disku, ale je v paměti, lze tak pomocí ní komunikovat v reálném čase mezi procesy, což jsem využil
			pro komunikace s laděným procesem. Na straně knihovny se vytvoří vlákno, které tyto soubory otevře a neustále z nich čte data. Umí také laděnému procesu
			zaslat data na jeho standardní vstup. Při zapnutí laděného procesu v GDB je poté jeho standardní výstup i vstup přesměrován do těchto souborů pomocí
			standardního přesměrování vstupu a výstupu ze shellu. % TODO: co je shell?
			
			
			Po ukočení laděného procesu se poté tyto dočasné soubory smažou.
		\end{description}
			
	\subsection{Vizualizační část}
	
	\subsection{Ovládání laděného procesu}
	\subsection{Vizualizace paměti procesu}

\section{Závěr}
\label{sec:Conclusion}
\subsection{Další vývoj}
IDE - kompilace, projekty, doplňování kódu
backendy pro více ladících nástrojů
rozšíření vizualizace


\bigskip
\begin{flushright}
Jakub Beránek
\end{flushright}

%\printbibliography
\printbibheading[title=Reference, heading=bibintoc]
\printbibliography

\appendix
\section{Grafická rozhraní debuggerů}
\label{appendix:gui}
\InsertFigure{Figures/appendix_gui_gdb_tui}{\textwidth}{Textové rozhraní GDB (TUI)}{fig:AppendixGuiGdbTui}
\InsertFigure{Figures/appendix_gui_ddd}{\textwidth}{Rozhraní programu DDD}{fig:AppendixGuiDDD}
\InsertFigure{Figures/appendix_gui_clion}{\textwidth}{Rozhraní programu CLion (zdroj: https://www.jetbrains.com/clion)}{fig:AppendixGuiClion}
%\InsertFigure{Figures/appendix_gui_gede}{\textwidth}{Rozhraní programu Gede (http://gede.acidron.com/)}{fig:AppendixGuiGede}


\clearpage

\end{document}
