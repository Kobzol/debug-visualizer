\documentclass[bc,male,java,dept460]{diploma}						% jednostranny dokument
%\documentclass[bc,male,java,dept460,twoside]{diploma}		% oboustranny dokument
\usepackage[czech]{babel}

% remark - poznámka
% definition - definice
% theorem - věta
% example - příklad
% \begin{lstlisting}[label=src:Java,caption=Program v jazyce Java]
% \lstinputlisting[label=src:JavaExternal,caption={Program v jazyce Java, načtený z externího souboru}]{MyClass.java}
% \InsertFigure{Figures/Obr1}{40mm}{Pokusný obrázek -- absolutní velikost}{fig:SampleFigAbs}
% \InsertFigure{Figures/Obr1}{0.7\textwidth}{Pokusný obrázek -- relativní velikost}{fig:SampleFigRel}
% \InsertSidewaysFigure{Figures/Obr1}{0.6\textheight}{Pokusný obrázek -- otočený naležato}{fig:SampleFigSideway}

\ThesisAuthor{Jakub Beránek}
\ThesisTitle{Vizualizace ladění aplikací}
\EnglishThesisTitle{Visualization of application debugging}

\SubmissionDate{29. dubna 2016}

\PrintPublicationAgreement{true}

\AccessRestriction{Zde vložte text dohodnutého omezení přístupu k Vaší práci, chránící například firemní know-how.
Zde vložte text dohodnutého omezení přístupu k Vaší práce, chránící například firemní know-how.
A zavazujete se, že\par
\begin{enumerate}
\item o práci nikomu neřeknete,
\item po obhajobě na ni zapomenete a
\item budete popírat její existenci.
\end{enumerate}
A ještě jeden důležitý odstavec. A ještě jeden důležitý odstavec.
A ještě jeden důležitý odstavec. A ještě jeden důležitý odstavec.
A ještě jeden důležitý odstavec. A ještě jeden důležitý odstavec.
Konec textu dohodnutého omezení přístupu k Vaší práci.}

\Thanks{Rád bych na tomto místě poděkoval všem, kteří mi s prací pomohli, protože bez nich by tato práce nevznikla.}

\CzechAbstract{Tato bakalářská práce se zabývá vizualizací ladění programů napsaných v jazyce C a C++. První část popisuje
obecné principy ladění programů. Jsou zde popsány konstrukce, které se při ladění používají, způsob, jakým ladící nástroje
provádějí ladění programů a také existující ladících nástrojy a jejich grafické nástavby. Druhá část popisuje možnosti komunikace
s ladícími nástroji. Je v ní popsána implementace grafického nástroje, který vizualizuje pamět a stav procesu během jeho ladění
za využití existujících ladících nástrojů  }

\CzechKeywords{ladění programů, vizualizace paměti}

\EnglishAbstract{This is English abstract. This is English abstract. This is English abstract. This is English abstract. This is English abstract. This is English abstract.}

\EnglishKeywords{typography, \LaTeX, master thesis}

\AddAcronym{GNU}{GNU's Not Unix!}
\AddAcronym{GDB}{The GNU Project Debugger}

% Zadame cestu a jmeno souboru ci nekolika souboru s digitalizovanou podobou zadani prace
% Pri sazbe se pak hledaji soubory Figures/Zadani1.jpg, Figures/Zadani2.jpg atd.
% Do diplomove prace se postupne vlozi vsechny existujici soubory Figures/ZadaniXXX.jpg
% Pokud toto makro zapoznamkujeme sazi se stranka s upozornenim
%\ThesisAssignmentImagePath{Figures/Zadani}

% Zadame soubor s digitalizovanou podobou prohlaseni
% Pokud toto makro zapoznamkujeme sazi se cisty text prohlaseni
\DeclarationImageFile{Figures/Prohlaseni.jpg}

% Zacatek dokumentu
\begin{document}

% Nechame vysazet titulni strany.
\MakeTitlePages

% Asi urcite budeme potrebovat obsah prace.
\tableofcontents
\cleardoublepage	% odstrankujeme, u jednostranneho dokumentu o jednu stranku, u oboustrenneho o dve

% Jsou v praci tabulky? Pokud ano vysazime jejich seznam.
\listoftables
\cleardoublepage	% odstrankujeme, u jednostranneho dokumentu o jednu stranku, u oboustrenneho o dve

% Jsou v praci obrazky? Pokud ano vysazime jejich seznam.
\listoffigures
\cleardoublepage	% odstrankujeme, u jednostranneho dokumentu o jednu stranku, u oboustrenneho o dve

% Jsou v praci vypisy programu? Pokud ano vysazime jejich seznam.
\lstlistoflistings
\cleardoublepage	% odstrankujeme, u jednostranneho dokumentu o jednu stranku, u oboustrenneho o dve

% Zacneme uvodem
\section{Úvod}

\section{Principy ladění programů}
	Ladění je nezbytná součást vývoje programů, která dovoluje programátorům detailně sledovat a ovládat průběh běžícího procesu a také
	číst a zapisovat jeho paměť. K tomuto slouží ladící nástroje, které vytváří asociaci mezi zdrojovým kódem a binárním spustitelným
	souborem a poskytují tak tvůrci kódu možnost ladit kód na vysoké úrovni abstrakce, tj. na úrovni samotného zdrojového kódu.
	Tato kapitola popisuje obecné principy fungování těchto ladících nástrojů, způsob mapování z binárních instrukcí do zdrojového kódu,
	ovládání běžícího procesu a běžné konstrukce používané při ladění. Konkrétně je popis zaměřen na programy napsané v jazycích C a C++
	v prostředí UNIXových systémů.

	\subsection{Obecné principy ladění programů}
	\subsection{Uspořádání paměti procesů}
	\subsection{Popis průběhu ladění}

\section {Existující ladící nástroje}
	\subsection{GDB}
	\subsection{LLDB}
	\subsection{Grafické nádstavby}

\section{Komunikace s ladícím nástrojem}
	\subsection{Využití Python API pro GDB}
	\subsection{Využití Python API pro LLDB}
	\subsection{Využití GDB MI protokolu}
	
\section{Implementace vizualizačního nástroje}
	\subsection{Ovládání laděného procesu}
	\subsection{Vizualizace paměti procesu}

\section{Závěr}
\label{sec:Conclusion}
\cite{goossens94,lamport94}.

\bigskip
\begin{flushright}
Jakub Beránek
\end{flushright}

\begin{thebibliography}{99}

\bibitem{goossens94} Goossens, Michel,
\textit{The \LaTeX\ companion,} New York: Addison, 1994.

\bibitem{lamport94} Lamport, Leslie,
\textit{\LaTeX: a document preparation system: user's guide and reference manual},
New York: Addison-Wesley Pub. Co., 1994.

\end{thebibliography}

\appendix
\clearpage

\end{document}
